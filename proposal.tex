\documentclass[demo, MAIN.tex]{subfiles}

\begin{document}

\noindent\textbf{RESEARCH PROPOSAL}

\bigskip

\noindent\begin{tabularx}{\textwidth}{|>{\columncolor{tcc}}X|}
  \hline In this section, you must incorporate the following aspects:
  \begin{enumerate}[label={\alph*)},nosep]
  \item Theoretical-conceptual and state-of-the-art developments that
    underpin the proposal
  \item Objectives and hypotheses or research questions
  \item Methodology
  \item Work Plan, schedule and Gantt chart
  \item Background information to assess the capacity of the team to
    implement the proposal
  \item Scientific or technological novelty of your proposal
  \end{enumerate}

  \medskip

  Remember that:
  \begin{itemize}
    \vspace{-0.3cm}
  \item[-] All text, paragraphs or textual phrases from a
    bibliographic reference, whether \underline{by other authors or their own},
    must be duly identified in the text and in the list of references.
    \vspace*{-0.6cm}
  \item[-] You must strictly comply with what is established in
    Bases Concurso Nacional de Proyectos Fondecyt Regular 2023.
  \end{itemize}
  
  \medskip
  \vspace{-0.3cm}
  This file must contain a maximum of \textbf{10 pages} (use letter size
  format, Verdana font size 10 or similar).\\  \hline
\end{tabularx}

%%%%%%%%%%%%%%%%%%%%%%%%%%%%%%%%%%%%%%%%%%%%%%%%%%%%%%%%%%%%%%%%%%%%%%%%%%%%%%%%
\section{State-of-art}

Explique detalladamente el problema o aplicación que busca
trabajar. Para acompañar su proyecto, puede utilizar ecuaciones dentro
del texto como $\frac{x}{\sqrt{1-x^2}}$, o bien ecuaciones enumeradas,
por ejemplo:
\begin{equation}\label{eq:euler}
\exp(i\theta) = \cos(\theta) + i\sin(\theta).
\end{equation}

Puede citar ecuaciones enumeradas, por ejemplo la ecuación
\eqref{eq:euler} es una de las ecuaciones básicas del cálculo
complejo.

También puede agregar figuras para explicar mejor sus ideas. Trate de
citarlas adecuadamente en el texto, por ejemplo, la figura
\ref{fig:figura-interesante} muestra un ejemplo usado en wikipedia
para explicar la electricidad estática~\cite{wikistatic}.

\begin{figure}[h!]
  \centering
  \includegraphics[width=0.5\textwidth]{img/static_cat_wikipedia.jpg}
  \caption{Bolas de poliestireno adheridos al pelaje de un gato debido
    a la electricidad estática.~\cite{wikistatic}.}
  \label{fig:figura-interesante}
\end{figure}

\subsection{A subsection}

Si cree necesario, puede separar el texto en subsecciones.

Para darle peso a su proyecto, puede citar libros, páginas web o artículos científicos. Por ejemplo, esta es una referencia 
\cite{AF:2003} o dos referencias juntas \cite{AF:2003, CEL:arXiv, MS}.

%%%%%%%%%%%%%%%%%%%%%%%%%%%%%%%%%%%%%%%%%%%%%%%%%%%%%%%%%%%%%%%%%%%%%%%%%%%%%%%%
\section{Goals}

\subsection{Main goal}

Un único párrafo indicando, en general, el objetivo de este proyecto o
los resultados esperados. Este párrafo debe empezar por un verbo (Por
ejemplo, resolver, estudiar, simular, derivar, entre otros).

\subsection{Specific goals}
\begin{enumerate}
\item Escribir un máximo de tres objetivos, todos distintos y deben empezar con un verbo.
\item Esta es una lista de objetivos más simples que ayudarán a cumplir el objetivo general.
\item Cada objetivo específico estará relacionado a una metodología.
\end{enumerate}

\section{Hypothesis}


\section{Methodology}

\begin{enumerate}
\item Describa cómo llevará a cabo cada objetivo específico. 
\item Cada actividad que proponga, debe ser listada en la carta Gantt (ver plan de trabajo).
\item Enliste las herramientas que utilizará para llevar a cabo su proyecto (por ejemplo, lenguajes de programación, recursos computacionales disponibles, características de los computadores donde llevará a cabo su trabajo, etc.).  
\end{enumerate}

\subsection{Methodology for specific goal \#1} 

\subsection{Methodology for specific goal \#2}

\subsection{Methodology for specific goal \#3}


\section{Work plan}

Explique las tareas que realizará para cumplir con los objetivos
específicos. Además, justifique el tiempo que cree que dedicará a cada
una de estas tareas. Luego, resumirá estas tareas y tiempos de
ejecución en una tabla que llamamos \textit{carta Gantt}.

\subsection{Gantt chart}
% example adapted from https://tex.stackexchange.com/questions/587422/draw-gantt-chart-latex-with-pgfgantt/587449#587449
\begin{center}
  \begin{ganttchart}[vgrid, hgrid, 
    y unit title=0.5cm,
    y unit chart=0.5cm,
    x unit=1.5cm,
    title height=1,
    progress label text={},
    bar height=0.5 ]{1}{8}   % divide la tabla en 8 columnas. 
    % Esta carta gantt está dividida en 8 semanas (2 meses). Cada
    % columna representa una semana.

    % periodos
    \gantttitle{2021}{8} \\ % titulo usa 8 columnas
    \gantttitle{Noviembre}{4} \gantttitle{Diciembre}{4} \\ % 2 titulos, cada uno usa 4 columnas

    % tareas
    \ganttbar[progress=30, name=obj1]{Actividad \#1}{1}{2} \\  % actividad se realiza entre las semanas 1-2; lleva 30% de progreso
    \ganttbar[progress=50           ]{Actividad \#2}{3}{5} \\  % actividad entre las semanas 3-5, 50% de progreso
    \ganttbar[progress=0, name=obj3 ]{Actividad \#3}{4}{7} \\ % actividad entre semanas 4-7, aun no inicia
    \ganttbar[progress=0            ]{Actividad \#4}{6}{8}               % actividad entre semanas 6-8, aun no inicia
    
    %relations 
    \ganttlink{obj1}{obj3} % relaciona las tareas que definen 'name' (obj3 depende de obj1)
  \end{ganttchart}
\end{center}


% If this file is compiled as a separate file, do not print the references list.
% TODO: Tried to use \nobibliography (from bibentry), but it is incompatible with the aiprev4-1 bib style
\ifSubfilesClassLoaded{%
  \newsavebox\mytempbib
  \savebox\mytempbib{\parbox{\textwidth}{\bibliography{references}}}
}{}

\end{document}
